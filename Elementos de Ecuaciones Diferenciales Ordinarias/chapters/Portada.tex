\begin{titlepage}
\begin{center}
\vspace*{+0.5in}
\textbf{\textsc{\begin{Huge}Elementos de Ecuaciones\vspace*{+0.1in} Diferenciales Ordinarias\end{Huge}}}

\vspace*{+0.15in}
\begin{figure}[htb]
\begin{center}
\includegraphics[width=8cm]{../imagenes/complu}\\\ \\\ \\
\textsc{\textbf{\begin{LARGE}Universidad Complutense de Madrid\end{LARGE}}}\\\ \\\ \\\ \\
\textsc{\begin{Large}Facultad de Ciencias Matemáticas\end{Large}}\\\ \\
\textsc{\begin{large}Doble Grado en Matemáticas e Ingeniería Informática\end{large}}
\end{center}
\end{figure}

\vspace*{0.25in}

\textsc{\textbf{\begin{large} Javier Pellejero\end{large}}\\}
Curso 2016-2017
\end{center}
\end{titlepage}
\blankpage

\pagenumbering{Roman} % para comenzar la numeracion de paginas en numeros romanos
	\chapter*{}
\begin{flushright}
\textit{Debemos dividir nuestro tiempo entre política y ecuaciones. Pero las ecuaciones son más importantes para mí, porque la política es para el momento actual y una ecuación es para la eternidad}.\\\ \\ Albert Einstein
\end{flushright}

	\chapter*{Prefacio}
	Aquí va el prefacio, evidentemente

	\tableofcontents
	\mainmatter % Empieza a numerar en números arábigos 
