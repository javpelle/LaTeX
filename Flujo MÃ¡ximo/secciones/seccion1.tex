\section{Introducción}
Cuando hablamos del flujo máximo de una red de flujo, estamos tratando el campo que busca la optimización de la afluencia entre vértices de un grafo, de manera que uno o varios nodos denominados sumideros reciba la mayor cantidad de dicha afluencia siempre de acuerdo a unas restricciones de capacidad entre los nodos del grafo. Es decir, limitando la capacidad de transporte de las aristas.\\

En definitiva, nos son dados unos vértices conectados o no por aristas a las que queremos asignar valores de flujo maximizando la recepción de ``mercancía'' de los vértices o nodos sumideros desde unos vértices iniciales a los que denominaremos fuentes.

\begin{defi} A continuación, definiremos una red de flujo, pues no es equivalente a un grafo cualquiera, sino que tiene unos requisitos muy concretos.
\begin{enumerate}[1)]
\item Una red de flujo es un \textbf{grafo dirigido} $G = (V,E)$ que cumple que para cada arista $(u,v)\in E$, su capacidad es mayor o igual que cero ($c(u,v) \geq 0$). Denotamos dicha propiedad como de \textbf{no negatividad}. Además, si $(u,v)\in E$ tenemos que $(v,u)\notin E$. Cabe destacar que normalizamos $c(u,v) = 0,\ \forall (u,v)\notin E$.
\item Definimos dos nodos fundamentales, el nodo \textbf{fuente} ($s$) y el nodo \textbf{sumidero} (t). Entre estos dos vértices existe al menos un camino formado por el resto de nodos de $V$; es más, para cada par de vértices de $V$ existe al menos un camino que los conecta: tenemos que el grafo $G$ es \textbf{conexo}. Por ser $G$ conexo obtenemos el resultado $\card(E)\geq \card(V)-1$, puesto que necesitamos al menos $n - 1$ aristas para unir todos los vértices de un grafo de $n$ componentes.\\  
El nodo fuente $s$ solo tiene aristas salientes, por lo que si $(u,v)\in E \implies v\neq s$, mientras que el nodo sumidero $t$ solo tiene aristas entrantes, así que si $(u,v)\in E \implies u\neq t$.
\item Definimos la \textbf{función de flujo} $\function{f}{V\times V}{\R}$ sobre el grafo $G$ que cumple:
\begin{itemize}
\item Para todo $u, v\in V$, se cumple que $0\leq f(u,v)\leq c(u,v)$. A esta propiedad la denominamos \textbf{restricción de capacidad}.
\item Si una arista $(u,v)\notin E$, entonces su función de flujo es $f(u,v)=0$.
\item Para todo $u\in V\setminus\{s,t\}$, tenemos que $\stackbin[v\in V]{}\sum f(v,u) = \stackbin[v\in V]{}\sum f(u,v)$; es decir, el flujo entrante en un nodo distinto de la fuente y el sumidero es el mismo que el flujo saliente. A esta propiedad la denominamos \textbf{conservación de flujo}.
\item Además, denotamos el valor $|f|= \stackbin[v\in V]{}\sum f(s,v)- \stackbin[v\in V]{}\sum f(v,s)$. $|f|$ es la diferencia entre el flujo saliente y el entrante de la fuente ($|\cdot|$ no denota ni cardinalidad ni valor absoluto). Aunque, en general, no tenemos aristas dirigidas hacia la fuente $\left(\stackbin[v\in V]{}\sum f(v,s)= 0\right)$, lo incluimos porque en ocasiones podemos tener redes residuales en las que el flujo dirigido a la fuente puede ser significante. Es importante destacar que $|f|= \stackbin[v\in V]{}\sum f(v,t)- \stackbin[v\in V]{}\sum f(t,v)$. Este resultado es evidente, puesto que tenemos que los vértices intermedios (distintos de $s\y t$) tienen mismo flujo saliente que entrante, por lo tanto el flujo saliente de la fuente debe ser el mismo que el entrante en el sumidero.
\end{itemize}
\end{enumerate}
\end{defi}

\begin{ejem} Veamos a continuación un ejemplo de red de flujo.\\
El grafo representado en la \textbf{Figura 1.1} se identifica con un problema de flujo máximo, donde cada arista dirigida contempla dos números. El primero de ellos se identifica con la función de flujo en $(u,v)$, es decir, $f(u,v)$; mientras que el segundo con la capacidad, $c(u,v)$.

Podemos ver que se cumplen las propiedades de \textit{restricción de capacidad} y de \textit{conservación de flujo}.
\begin{figura}\ \begin{center}\definecolor{zzttqq}{rgb}{0.15,0.35,0.15}

\begin{tikzpicture}[x=1.5cm, y=1.5cm]
	%\fill (-3.2,0) circle (0.1pt)node[anchor=east] {$20$};
	%\fill (3.2,0) circle (0.1pt)node[anchor=west] {$-20$};
    \node[circle,draw=red,fill=red!20!] (v1) at (-1.5,1.5) {$v_1$};
    \node[circle,draw=red,fill=red!20!] (v3) at (1.5,1.5) {$v_3$};
    \node[circle,draw=red,fill=red!20!] (t) at (3,0) {$t$};
    \node[circle,draw=red,fill=red!20!] (v4) at (1.5,-1.5) {$v_4$};
    \node[circle,draw=red,fill=red!20!] (v2) at (-1.5,-1.5) {$v_2$};
    \node[circle,draw=red,fill=red!20] (s) at (-3,0) {$s$};
    \draw[color=zzttqq, ultra thick, -latex]  (s) edge node[rotate = 45, above,color=black]{11/16} (v1);
	\draw[color=zzttqq, ultra thick, -latex]  (s) edge node[rotate = -45, below,color=black]{8/13} (v2);
	\draw[-latex, color=zzttqq, ultra thick]  (v2) edge node[right,color=black]{1/4} (v1);
	\draw[-latex, color=zzttqq, ultra thick]  (v1) edge node[above,color=black]{12/12} (v3);
	\draw[-latex, color=zzttqq, ultra thick]  (v2) edge node[above,color=black]{11/14} (v4);
	\draw[-latex, color=zzttqq, ultra thick]  (v3) edge node[rotate=45,above,color=black]{4/9} (v2);
	\draw[-latex, color=zzttqq, ultra thick]  (v4) edge node[right,color=black]{7/7} (v3);
	\draw[-latex, color=zzttqq, ultra thick]  (v3) edge node[rotate=-45,above,color=black]{15/20} (t);
	\draw[-latex, color=zzttqq, ultra thick]  (v4) edge node[rotate=45,below,color=black]{4/4} (t);
\end{tikzpicture}\end{center}\end{figura}

En particular, $0\leq f(u,v)\leq c(u,v)\ \forall (u,v)\in E$, y además, analizando\\
$\{v_1,v_2,v_3,v_4\}=V\setminus \{s,t\}$ tenemos:
\begin{itemize}
\item En $v_1$ tenemos la arista saliente $(v_1,v_3)$ y las entrantes $(s,v_1)\y(v_2,v_1)$ que cumplen $f(v_1,v_3)=12 = f(s,v_1)+f(v_2,v_1)$.
\item En $v_2$ tenemos las aristas salientes $(v_2,v_1)\y(v_2,v_4)$ y las entrantes $(s,v_2)\y(v_3,v_2)$ que cumplen
$f(v_2,v_1)+f(v_2,v_4)=12=f(s,v_2)+f(v_3,v_2)$.
\item En $v_3$ tenemos las aristas salientes $(v_3,t)\y(v_3,v_2)$ y las entrantes $(v_1,v_3)\y(v_4,v_3)$ que cumplen
$f(v_3,t)+f(v_3,v_2)=19=f(v_1,v_3)+f(v_4,v_3)$.
\item En $v_4$ tenemos las aristas salientes $(v_4,v_3)\y(v_4,t)$ y la entrante $(v_2,v_4)$ que cumplen
$f(v_4,v_3)+f(v_4,t)=11=f(v_2,v_4)$.
\end{itemize}
Por último señalamos que $|f| = 19$.
\end{ejem}

Para finalizar esta introducción plantearemos un problema basado en la \textbf{Figura 1.1}.\\

\begin{ejem} Los regantes murcianos, aquejados de su continua sequía, no tienen con qué regar sus preciados latifundios y campos de golf en medio del árido desierto. Para resolver este problema deciden robar ingentes cantidades de agua a la humilde población castellano-manchega. El embalse de Buendía, situado entre Cuenca y Guadalajara y de gran capacidad aunque muy mermado por los constantes saqueos murcianos, parece un lugar ideal del que abastecerse. Para hacer llegar el agua a las calurosas tierras murcianas se dispone de diversos embalses y pantanos conectados por trasvases cuyos caudales diarios en cierta unidad son conocidos. Además tenemos las siguientes conexiones:
\begin{itemize}
\item Buendía está conectado a los embalses de Entrepeñas y Beleña.
\item A su vez, Entrepeñas puede mandar agua exclusivamente al pantano de Alarcón.
\item Beleña puede enviar agua a Entrepeñas y al embalse de la Fuensanta.
\item Alarcón tiene trasvases hacia Beleña y hasta el destino final: Valdeinfierno (Murcia).
\item Por último, el embalse de la Fuensanta envía agua a Alarcón y a Valdeinfierno.
\end{itemize}
En la \textbf{Figura 1.2} están representados los embalses y pantanos, los trasvases y los caudales de los mismos. En las siguientes secciones contemplaremos algoritmos que maximizan la solución a estos problemas, es decir, nos permitirán calcular cuál es el caudal máximo diario que puede llegar a Murcia. Adelantamos que la solución máxima a este problema (que no es complicado calcular ``a ojo'') es:\\
$f(s,v_1)=12,\ f(s,v_2) = 11,\ f(v_1, v_3) = 12,\ f(v_2,v_1) = 0,\ f(v_2,v_4) = 11,\ f(v_3,v_2)=0,\\
f(v_3,t)=19,f(v_4,v_3)=7\y f(v_4,t)=4$. Por lo que podemos llevar a Murcia un máximo de 23 unidades de agua diarias.
\begin{figura}\ \begin{center}\definecolor{zzttqq}{rgb}{0.15,0.35,0.15}

\begin{tikzpicture}[x=1.5cm, y=1.5cm]
	\fill (-3.2,0) circle (0.1pt)node[anchor=east] {Buendía};
	\fill (3.2,0) circle (0.1pt)node[anchor=west] {Valdeinfierno};
	\fill (-1.5,1.8) circle (0.1pt)node[anchor=south] {Entrepeñas};
	\fill (-1.5,-1.8) circle (0.1pt)node[anchor=north] {Beleña};
	\fill (1.5,1.8) circle (0.1pt)node[anchor=south] {Alarcón};
	\fill (1.5,-1.8) circle (0.1pt)node[anchor=north] {Fuensanta};
    \node[circle,draw=red,fill=red!20!] (v1) at (-1.5,1.5) {$v_1$};
    \node[circle,draw=red,fill=red!20!] (v3) at (1.5,1.5) {$v_3$};
    \node[circle,draw=red,fill=red!20!] (t) at (3,0) {$t$};
    \node[circle,draw=red,fill=red!20!] (v4) at (1.5,-1.5) {$v_4$};
    \node[circle,draw=red,fill=red!20!] (v2) at (-1.5,-1.5) {$v_2$};
    \node[circle,draw=red,fill=red!20] (s) at (-3,0) {$s$};
    \draw[color=zzttqq, ultra thick, -latex]  (s) edge node[rotate = 45, above,color=black]{16} (v1);
	\draw[color=zzttqq, ultra thick, -latex]  (s) edge node[rotate = -45, below,color=black]{13} (v2);
	\draw[-latex, color=zzttqq, ultra thick]  (v2) edge node[right,color=black]{4} (v1);
	\draw[-latex, color=zzttqq, ultra thick]  (v1) edge node[above,color=black]{12} (v3);
	\draw[-latex, color=zzttqq, ultra thick]  (v2) edge node[above,color=black]{14} (v4);
	\draw[-latex, color=zzttqq, ultra thick]  (v3) edge node[rotate=45,above,color=black]{9} (v2);
	\draw[-latex, color=zzttqq, ultra thick]  (v4) edge node[right,color=black]{7} (v3);
	\draw[-latex, color=zzttqq, ultra thick]  (v3) edge node[rotate=-45,above,color=black]{20} (t);
	\draw[-latex, color=zzttqq, ultra thick]  (v4) edge node[rotate=45,below,color=black]{4} (t);
\end{tikzpicture}\end{center}\end{figura}
\end{ejem}