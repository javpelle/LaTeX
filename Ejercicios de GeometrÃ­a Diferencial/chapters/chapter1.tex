\chapter{Curvas parametrizadas y longitud de un arco de curva}

\begin{ejercicio} Hallar una curva parametrizada $\alpha$ cuya traza es el círculo $x^2+y^2=1$, con $\alpha(t)$ recorriéndolo en el sentido de las agujas del reloj y con $\alpha(0) =(0,1)$.\\

Una solución a este ejercicio $\alpha(t)=(\sin t, \cos t)$. Es claro que $\alpha(0) =(\sin 0,\cos 0) =(0,1)$ y que al avanzar, por ejemplo a $\alpha(\frac{\pi}{2})=(\sin \frac{\pi}{2},\cos \frac{\pi}{2}) = (1,0)$ es en el sentido de las agujas del reloj.
\end{ejercicio}

\begin{ejercicio} Sea $\alpha(t)$  una curva que no pasa por el origen. Si $\alpha(t_0)$ es el punto de la traza de $\alpha$ más cercano al origen y $\alpha'(t_0)\neq 0$, demostrar que el vector posición $\alpha(t_0)$ es ortogonal a $\alpha'(t_0)$.\\

Definimos la norma
\end{ejercicio}

\begin{ejercicio} Sea $\ait$ una curva y $v\in\R^3$ un vector dado. Si $\alpha'(t)$ es ortogonal a $v$ para todo $t\in I$, y si $\alpha(0)$ también lo es, demuestre que $\alpha(t)$ es ortogonal a $v$ para todo $t\in I$.
\end{ejercicio}

\begin{ejercicio} Si $\ait$ es una curva regular, demuestre que $|\alpha(t)|$ es constante (diferente de cero) si y sólo si $\alpha(t) \perp \alpha'(t)$ para todo $t\in I$.
\end{ejercicio}

\begin{ejercicio} Si $\ait$ es una curva, y $\function{M}{\R^3}{\R^3}$ es un movimiento rígido, demostrar que las longitudes de $\alpha$ y $M\circ\alpha$ entre $a\y b$ coinciden.
\end{ejercicio}

\begin{ejercicio} Demuestre que las líneas tangentes a la curva $\alpha(t)=(3t,3t^2,2t^3)$ forman un ángulo constante con la recta $y=0,\ z=x$.
\end{ejercicio}