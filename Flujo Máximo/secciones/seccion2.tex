\section{Modelación de problemas de flujo}

En esta sección vamos a tratar de solucionar y normalizar algunos casos de problemas de flujo que no cumplan todas y cada una de las especificaciones dadas en la anterior sección.
\subsection{Modelación de redes con aristas antiparalelas}

Decimos que dos aristas $(u,v)\y (v,u)$ son antiparalelas si ambas pertenecen a $E$, es decir, son paralelas pero con sentidos opuestos. En la \textit{sección 1} indicamos que si $(u,v)\in E$ entonces $(v,u)\notin E$. Para solucionar este inconveniente tomaremos un vértice auxiliar $v'$ y convertiremos el grafo $G=(V,E)$ en el grafo $G'=(V',E')$.\\

Tenemos que $V'=V\cup \{v'\}$, veamos como redefinimos $E'$:\\
Puesto que $(u,v)\y (v,u)\in E$, una de estas dos debe ``desaparecer''. Sin pérdida de generalidad quitaremos $(u,v)$ para añadir $(u,v')\y(v',v)$ de tal manera que\\
$c(u,v')=c(v',v)=c(u,v)$ (a la anterior $c(u,v)$ puesto que al quitarla esta pasa a ser cero). De esta manera hemos tendido un puente auxiliar entre $u$ y $v$ y ahora $G'$ ya está listo para poder trabajar sobre él como un problema de flujo máximo. Así, $E'$ ha quedado como $E'=\left(E\setminus\left\{(u,v)\right\}\right)\cup\left\{(u,v'),(v',v)\right\}$.\\

\begin{ejem} Veamos un ejemplo gráfico de unos posibles $G$ con aristas antiparalelas y su normalización y $G'$.

\begin{figura}\ \begin{center}\definecolor{zzttqq}{rgb}{0.15,0.35,0.15}

\begin{tikzpicture}[x=1.5cm, y=1.5cm]
	%\fill (-3.2,0) circle (0.1pt)node[anchor=east] {$20$};
	%\fill (3.2,0) circle (0.1pt)node[anchor=west] {$-20$};
    \node[circle,draw=red,fill=red!20!] (v1) at (-1.5,1.5) {$v_1$};
    \node[circle,draw=red,fill=red!20!] (v3) at (1.5,1.5) {$v_3$};
    \node[circle,draw=red,fill=red!20!] (t) at (3,0) {$t$};
    \node[circle,draw=red,fill=red!20!] (v4) at (1.5,-1.5) {$v_4$};
    \node[circle,draw=red,fill=red!20!] (v2) at (-1.5,-1.5) {$v_2$};
    \node[circle,draw=red,fill=red!20] (s) at (-3,0) {$s$};
    \draw[color=zzttqq, ultra thick, -latex]  (s) edge node[rotate = 45, above,color=black]{16} (v1);
	\draw[color=zzttqq, ultra thick, -latex]  (s) edge node[rotate = -45, below,color=black]{13} (v2);
	%\draw[-latex, color=zzttqq, ultra thick]  (v2) edge node[auto = right][right,color=black]{4} (v1);
	%\draw[-latex, color=zzttqq, ultra thick]  (v1) edge node[auto = left][left,color=black]{10} (v2);
	\draw[-latex, color=zzttqq, ultra thick]  (v1) edge node[above,color=black]{12} (v3);
	\draw[-latex, color=zzttqq, ultra thick]  (v2) edge node[above,color=black]{14} (v4);
	\draw[-latex, color=zzttqq, ultra thick]  (v3) edge node[rotate=45,above,color=black]{9} (v2);
	\draw[-latex, color=zzttqq, ultra thick]  (v4) edge node[right,color=black]{7} (v3);
	\draw[-latex, color=zzttqq, ultra thick]  (v3) edge node[rotate=-45,above,color=black]{20} (t);
	\draw[-latex, color=zzttqq, ultra thick]  (v4) edge node[rotate=45,below,color=black]{4} (t);
	
	 \DoubleLine{v1}{v2}{latex-, color=zzttqq, ultra thick}{4}{-latex, color=zzttqq, ultra thick}{10}
\end{tikzpicture}\end{center}\end{figura}
\ \\

En la \textbf{Figura 2.1} podemos observar un grafo al que denotaremos por $G=(V,E)$, en el que tenemos dos aristas antiparalelas: $(v_1,v_2)\y(v_2,v_1)$, con capacidades\\
$c(v_1,v_2)=10\y(v_2,v_1)=4$. Para solucionar este hecho incluimos en $V$ el vértice $v'$ dando lugar a $V'$. Tomamos la arista $(v_1,v_2)$ y la sustituimos por las aristas $(v_1,v')\y(v',v_2)$ ambas de capacidad $10$ dando lugar a $E'$.
Así ya tenemos el nuevo grafo $G'$ compuesto por $(V',E')$ que podemos observar en la \textbf{Figura 2.2}.

\begin{figura}\ \begin{center}\definecolor{zzttqq}{rgb}{0.15,0.35,0.15}

\begin{tikzpicture}[x=1.5cm, y=1.5cm]
	%\fill (-3.2,0) circle (0.1pt)node[anchor=east] {$20$};
	%\fill (3.2,0) circle (0.1pt)node[anchor=west] {$-20$};
    \node[circle,draw=red,fill=red!20!] (v1) at (-1.5,1.5) {$v_1$};
    \node[circle,draw=red,fill=red!20!] (v3) at (1.5,1.5) {$v_3$};
    \node[circle,draw=red,fill=red!20!] (t) at (3,0) {$t$};
    \node[circle,draw=red,fill=red!20!] (v4) at (1.5,-1.5) {$v_4$};
    \node[circle,draw=red,fill=red!20!] (v2) at (-1.5,-1.5) {$v_2$};
    \node[circle,draw=red,fill=red!20] (s) at (-3,0) {$s$};
    \node[circle,draw=red,fill=red!20] (v5) at (-2.2,0) {$v'$};
    \draw[color=zzttqq, ultra thick, -latex]  (s) edge node[rotate = 45, above,color=black]{116} (v1);
	\draw[color=zzttqq, ultra thick, -latex]  (s) edge node[rotate = -45, below,color=black]{13} (v2);
	\draw[-latex, color=zzttqq, ultra thick]  (v2) edge node[right,color=black]{4} (v1);
	\draw[-latex, color=zzttqq, ultra thick]  (v1) edge node[above,color=black]{12} (v3);
	\draw[-latex, color=zzttqq, ultra thick]  (v1) edge node[rotate = 70,below,color=black]{10} (v5);
	\draw[-latex, color=zzttqq, ultra thick]  (v5) edge node[rotate = -70,above,color=black]{10} (v2);
	\draw[-latex, color=zzttqq, ultra thick]  (v2) edge node[above,color=black]{14} (v4);
	\draw[-latex, color=zzttqq, ultra thick]  (v3) edge node[rotate=45,above,color=black]{9} (v2);
	\draw[-latex, color=zzttqq, ultra thick]  (v4) edge node[right,color=black]{7} (v3);
	\draw[-latex, color=zzttqq, ultra thick]  (v3) edge node[rotate=-45,above,color=black]{20} (t);
	\draw[-latex, color=zzttqq, ultra thick]  (v4) edge node[rotate=45,below,color=black]{4} (t);
\end{tikzpicture}\end{center}\end{figura}
\end{ejem}

\subsection{Modelación de redes con múltiples fuentes y sumideros}

Una red de un problema de flujo máximo puede tener varias fuentes y sumideros. En el ejemplo de los trasvases, es altamente probable que el agua provenga originalmente de más de un embalse fuente y llegue hasta distintos embalses, pantanos y campos de regadío sumidero murcianos. Tenemos así un problema con nuestra definición original puesto que definimos una única fuente y un único sumidero.\\

En la \textbf{Figura 2.3} tenemos un ejemplo de un grafo con 5 fuentes, representadas por $s_1,s_2,s_3,s_4,s_5$ y 3 sumideros, $t_1,t_2,t_3$
\begin{figura}\ \begin{center}\input{figuras/figura2.3.tex}\end{center}\end{figura}

Para resolver este inconveniente basta con añadir dos vértices auxiliares a los que denotaremos $s$ y $t$ que denotaremos \textbf{superfuente} y \textbf{supersumidero} respectivamente. Tenemos así un nuevo conjunto de vértices $V'=V\cup \{s,t\}$ y nos falta añadir ciertas aristas a $E$ para normalizar nuestro problema.

Nuestro nuevo $E'$ estará compuesto por $E'=E\cup\{(s,s_i)\ |\ s_i\in E\}\cup\{(t_i,t)\ |\ t_i\in E\}$ de manera que las capacidades de estas aristas sean $c(s,s_i)=c(t_i,t)=\infty$. Hemos construido así puentes entre nuestras fuentes y la superfuente auxiliar y entre nuestros sumideros y el supersumidero auxiliar con aristas de capacidad infinita.\\

Intuitivamente, cualquier movimiento de flujo en el nuevo grafo $G'=(V',E')$ se comporta igual que en $G=(V,E)$. $s$ sólo provee la suma de flujo que ya proveían los distintos $s_i$ y $t$ sólo ``engulle'' la suma de lo que ya ``engullían'' los $t_i$.\\

En la \textbf{Figura 2.4} podemos observar cómo queda el grafo de la \textbf{Figura 2.1} tras la adición de la superfuente y el supersumidero y sus correspondientes aristas.
\begin{figura}\ \begin{center}\definecolor{zzttqq}{rgb}{0.15,0.35,0.15}

\begin{tikzpicture}[x=1.5cm, y=1.5cm]
	\node[circle,draw=red,fill=red!20] (s) at (0,4.25) {$s$};
	\node[circle,draw=red,fill=red!20] (s1) at (-3,3) {$s_1$};
	\node[circle,draw=red,fill=red!20] (s2) at (-1.5,3) {$s_2$};
	\node[circle,draw=red,fill=red!20] (s3) at (0,3) {$s_3$};
	\node[circle,draw=red,fill=red!20] (s4) at (1.5,3) {$s_4$};
	\node[circle,draw=red,fill=red!20] (s5) at (3,3) {$s_5$};
    \node[circle,draw=red,fill=red!20!] (v1) at (-2.25,1.75) {$v_1$};
    \node[circle,draw=red,fill=red!20!] (v2) at (-0.75,1.75) {$v_2$};
    \node[circle,draw=red,fill=red!20!] (v3) at (0.75,1.75) {$v_3$};
    \node[circle,draw=red,fill=red!20!] (v4) at (2.25,1.75) {$v_4$};
    \node[circle,draw=red,fill=red!20!] (t1) at (-1.5,0.5) {$t_1$};
    \node[circle,draw=red,fill=red!20!] (t2) at (0,0.5) {$t_2$};
    \node[circle,draw=red,fill=red!20!] (t3) at (1.5,0.5) {$t_3$};
    \node[circle,draw=red,fill=red!20!] (t) at (0,-0.75) {$t$};
    
	\draw[color=zzttqq, ultra thick, -latex]  (s) edge node[rotate = 20, above,color=black]{$\infty$} (s1);
	\draw[color=zzttqq, ultra thick, -latex]  (s) edge node[rotate = 40, above,color=black]{$\infty$} (s2);   
	\draw[color=zzttqq, ultra thick, -latex]  (s) edge node[right,color=black]{$\infty$} (s3);			
	\draw[color=zzttqq, ultra thick, -latex]  (s) edge node[rotate = -40, above,color=black]{$\infty$} (s4);
	\draw[color=zzttqq, ultra thick, -latex]  (s) edge node[rotate = -20, above,color=black]{$\infty$} (s5);
	
    \draw[color=zzttqq, ultra thick, -latex]  (s1) edge node[rotate = -60, above,color=black]{10} (v1);
    \draw[color=zzttqq, ultra thick, -latex]  (s2) edge node[rotate = 60, below,color=black]{12} (v1);
    \draw[color=zzttqq, ultra thick, -latex]  (s2) edge node[rotate = -60, above,color=black]{5} (v2);
    \draw[color=zzttqq, ultra thick, -latex]  (s3) edge node[rotate = 60, below,color=black]{8} (v2);
    \draw[color=zzttqq, ultra thick, -latex]  (s3) edge node[rotate = -60, above,color=black]{14} (v3);
    \draw[color=zzttqq, ultra thick, -latex]  (s4) edge node[rotate = 60, below,color=black]{7} (v3);
    \draw[color=zzttqq, ultra thick, -latex]  (s4) edge node[rotate = -60, above,color=black]{11} (v4);
    \draw[color=zzttqq, ultra thick, -latex]  (s5) edge node[rotate = 60, below,color=black]{2} (v4);
	
	\draw[color=zzttqq, ultra thick, -latex]  (v1) edge node[rotate = -60, above,color=black]{3} (t1);
    \draw[color=zzttqq, ultra thick, -latex]  (v2) edge node[rotate = 60, below,color=black]{15} (t1);
    \draw[color=zzttqq, ultra thick, -latex]  (v2) edge node[rotate = -60, above,color=black]{6} (t2);
    \draw[color=zzttqq, ultra thick, -latex]  (v3) edge node[rotate = 60, below,color=black]{20} (t2);
    \draw[color=zzttqq, ultra thick, -latex]  (v3) edge node[rotate = -60, above,color=black]{13} (t3);
    \draw[color=zzttqq, ultra thick, -latex]  (v4) edge node[rotate = 60, below,color=black]{18} (t3);
    
    \draw[color=zzttqq, ultra thick, -latex]  (t1) edge node[rotate = -45, above,color=black]{$\infty$} (t);
	\draw[color=zzttqq, ultra thick, -latex]  (t2) edge node[right,color=black]{$\infty$} (t);   
	\draw[color=zzttqq, ultra thick, -latex]  (t3) edge node[rotate = 45,below,color=black]{$\infty$} (t);	
	
\end{tikzpicture}\end{center}\end{figura}