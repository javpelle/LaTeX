\documentclass[11pt, oneside]{book}


\usepackage[utf8]{inputenc}
\usepackage[spanish]{babel}

%--------Codigos para la caligrafia, tipos de letras%---------------
\usepackage[T1]{fontenc}
%\usepackage[bitstream-charter]{mathdesign} 
%\usepackage{textcomp} %Paquete para algunos caracteres especiales

\usepackage{afterpage}
\usepackage{mathtools}

\usepackage{amssymb}
\usepackage[usenames]{color}
\usepackage{enumerate}
\usepackage{stackrel}
\usepackage{amsthm}
\usepackage{centernot}
\usepackage[a4paper, total={6in, 8in}]{geometry}
\usepackage{tikz}
\usetikzlibrary{arrows,matrix,positioning}

\theoremstyle{definition} % Cambio del estilo de los teoremas a normal
\newtheorem{ejercicio}{Ejercicio}


\newcommand{\deter}[1]{\left| \begin{matrix} #1	\end{matrix}\right|}

\newcommand{\doubleright}[2]{ \left. \begin{array}{ll}	#1 \\	#2 \\ \end{array} 	\right\} }
\newcommand{\doubleleft}[2]{ \left\{\begin{array}{ll}	#1 \\	#2 \\  \end{array}	\right. }
\newcommand{\doubleleftright}[2]{ \left\{\begin{array}{ll} #1 \\ #2 \\ \end{array} \right\}}
\newcommand{\double}[2]{ \left. \begin{array}{ll}	#1 \\	#2 \\	 \end{array} \right. }	
\newcommand{\tripleright}[3]{ \left. \begin{array}{ll}	#1 \\#2 \\#3\\	\end{array} \right\} }
\newcommand{\tripleleft}[3]{ \left\{ \begin{array}{ll} #1 \\#2 \\#3\\ \end{array} \right. }
\newcommand{\triple}[3]{ \left. \begin{array}{ll}	#1 \\#2 \\#3\\	\end{array} 	\right. }
\newcommand{\rg}{\mathrm{rg}}
\newcommand{\y}{\mathrm{\ y\ }}


\title{Resolución y discusión de sistemas lineales mediante sistemas matriciales y determinantes}


\usepackage{fancyhdr}
\pagestyle{fancy}
\rhead{Javier Pellejero Ortega}


\author{Javier Pellejero Ortega}


\begin{document}

\begin{ejercicio} ¿Cuántas veces más grande tiene que ser la arista de un tetraedro con respecto a la de un icosaedro que tienen el mismo área?
\end{ejercicio}

\begin{ejercicio} Suponiendo que la Tierra es una esfera perfecta y que el ecuador mide 40.075 kilómetros. ¿Cuál es su volumen?
\end{ejercicio}

\begin{ejercicio} Sea una circunferencia con centro en $C=(3,0)$. Realizamos un giro de 90 grados con centro en en $O=(0,0)$. ¿Dónde está el centro $C'$ de la circunferencia resultante? ¿Cuál es el vector $\vec{u}$ que usaríamos para hacer la traslación de $C$ a $C'$? Nota: Creo que el libro no os muestra las herramientas para calcular un giro analíticamente, así que tendréis que hacerlo gráficamente, de todas maneras se puede ver de cabeza.
\end{ejercicio}

\begin{ejercicio} Sean $\alpha\y\beta=360º -\alpha$ los ángulos correspondietes a dos giros con mismo centro $O$. Sea un punto cualquiera $P$ del plano, ¿dónde estará el punto resultante $P'$ de aplicar el producto de los giros $\alpha\y\beta$  a $P$? Nota: es un ejercicio general, no doy coordenadas así que no podéis resolverlo gráficamente (es decir, la solución tiene que ser analítica). Sin embargo, si tenéis dudas, hacer un ejemplo gráfico puede ayudar bastante.
\end{ejercicio}

\begin{ejercicio} Definimos el \textbf{producto escalar} de dos vectores $\vec u=(u_1,u_2),\ \vec{v}=(v_1,v_2)$ como $\vec{u}\cdot\vec{v}=u_1\cdot v_1 + u_2\cdot v_2$. Decimos que $\vec{u}$ es perpendicular a $\vec{v}$ si son no nulos (es decir, ninguno son el $(0,0)$) y $\vec{u}\cdot\vec{v}=0$. Encontrar un vector $\vec{v}$ perpendicular a $\vec{u}=(1,3)$.
\end{ejercicio}

Todos los ejercicios se pueden hacer en una o dos líneas, pero eso no implica que sean sencillos. De hecho no os preocupéis si no los sacáis, con estas cuestiones quiero ver hasta que punto entendéis los dos temas más allá de aprenderse una serie de fórmulas y plasmarlas en el examen, cosa que es posible sin entender lo que estáis haciendo (o al menos sin entenderlo del todo).

6 minutos por ejercicio (para un total de 30) debería ser suficiente para hacer el examen completo. Si no lo acabáis en ese tiempo preguntadme por la soluciones. Lo podéis hacer mirando el libro o estudiando previamente, como veais.

\end{document}