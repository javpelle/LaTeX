\documentclass[11pt, oneside]{book}


\usepackage[utf8]{inputenc}
\usepackage[spanish]{babel}

%--------Codigos para la caligrafia, tipos de letras%---------------
\usepackage[T1]{fontenc}
%\usepackage[bitstream-charter]{mathdesign} 
%\usepackage{textcomp} %Paquete para algunos caracteres especiales

\usepackage{afterpage}
\usepackage{mathtools}

\usepackage{amssymb}
\usepackage[usenames]{color}
\usepackage{enumerate}
\usepackage{stackrel}
\usepackage{amsthm}
\usepackage{centernot}
\usepackage[a4paper, total={6in, 8in}]{geometry}
\usepackage{tikz}
\usetikzlibrary{arrows,matrix,positioning}

\theoremstyle{definition} % Cambio del estilo de los teoremas a normal
\newtheorem{ejem}{Ejemplo}


\newcommand{\deter}[1]{\left| \begin{matrix} #1	\end{matrix}\right|}

\newcommand{\doubleright}[2]{ \left. \begin{array}{ll}	#1 \\	#2 \\ \end{array} 	\right\} }
\newcommand{\doubleleft}[2]{ \left\{\begin{array}{ll}	#1 \\	#2 \\  \end{array}	\right. }
\newcommand{\doubleleftright}[2]{ \left\{\begin{array}{ll} #1 \\ #2 \\ \end{array} \right\}}
\newcommand{\double}[2]{ \left. \begin{array}{ll}	#1 \\	#2 \\	 \end{array} \right. }	
\newcommand{\tripleright}[3]{ \left. \begin{array}{ll}	#1 \\#2 \\#3\\	\end{array} \right\} }
\newcommand{\tripleleft}[3]{ \left\{ \begin{array}{ll} #1 \\#2 \\#3\\ \end{array} \right. }
\newcommand{\triple}[3]{ \left. \begin{array}{ll}	#1 \\#2 \\#3\\	\end{array} 	\right. }
\newcommand{\rg}{\mathrm{rg}}


\title{Resolución y discusión de sistemas lineales mediante sistemas matriciales y determinantes}


\usepackage{fancyhdr}
\pagestyle{fancy}
\rhead{Javier Pellejero Ortega}


\author{Javier Pellejero Ortega}


\begin{document}

\begin{ejem}\ \\
$\left. \begin{matrix} x & + & y & + & z & = &11\\
2x & - & y & + & z & = &5\\
3x & + & 2y & + & z & = &24 \end{matrix}\right\}\longrightarrow A^*= \begin{pmatrix} 1&1&1&|&11\\2&-1&1&|&5\\3&2&1&|&24\end{pmatrix}$ Calculamos el det. de $A$.\\
$|A|=\deter{1&1&1\\2&-1&1\\3&2&1}=-1+3+4-(-3+2+2)=5\neq 0\implies \rg(A)=3=\rg(A^*)=$ nº incógnitas $\implies$ SCD.\\
Podemos resolver por métodos habituales (sustitución, igualación o reducción), por Gauss, por Cramer o matricialmente (por la inversa).
\begin{itemize}
\item Gauss\\
$\begin{pmatrix} 1&1&1&|&11\\2&-1&1&|&5\\3&2&1&|&24\end{pmatrix}\sim\begin{pmatrix} 1&1&1&|&11\\0&-3&-1&|&-17\\0&-1&-2&|&-11\end{pmatrix}\sim\begin{pmatrix} 1&1&1&|&11\\0&3&1&|&23\\0&-3&-6&|&-33\end{pmatrix}\sim\\
\begin{pmatrix} 1&1&1&|&11\\0&3&1&|&17\\0&0&-5&|&-10\end{pmatrix}$
Luego $\triple{x+y+z=11\implies x+5+2=11\implies x=4}{3y+z=17\implies 3y+2=17\implies 3y=15\implies y= 5}{-5z=-10\implies z=2}$
\item Cramer\\
$x=\dfrac{|A_x|}{|A|}=\dfrac{\deter{11&1&1\\5&-1&1\\24&2&1}}{5}=\dfrac{20}{5}=4\ \ \ y=\dfrac{|A_y|}{|A|}=\dfrac{\deter{1&11&1\\2&5&1\\3&24&1}}{5}=\dfrac{25}{5}=5\\
z=\dfrac{|A_z|}{|A|}=\dfrac{\deter{1&1&11\\2&-1&5\\3&2&24}}{5}=\dfrac{10}{5}=2$
\item Matricialmente\\
Tenemos que $AX=B$ donde $A$ sigue siendo nuestra matriz de coeficientes\\
$A=\begin{pmatrix} 1&1&1\\2&-1&1\\3&2&1\end{pmatrix}$, $X=\begin{pmatrix}x\\y\\z\end{pmatrix}$ y $B=\begin{pmatrix}11\\5\\24\end{pmatrix}$. Luego si $AX=B\implies X=A^{-1}B$
Calculamos la inversa mediante la fórmula $A^{-1}=\dfrac{(\mathrm{adj\ }A)^t}{|A|}$ (me ahorro este paso) y obtenemos que $A^{-1}=\dfrac{1}{5}\begin{pmatrix}-3&1&2\\1&-2&1\\7&1&-3\end{pmatrix}$ y por último\\
$\begin{pmatrix}x\\y\\z\end{pmatrix}=\dfrac{1}{5}\begin{pmatrix}-3&1&2\\1&-2&1\\7&1&-3\end{pmatrix}\begin{pmatrix}11\\5\\24\end{pmatrix}=\begin{pmatrix}4\\5\\2\end{pmatrix}$
Con lo que $x=4,\ y=5$ y $z=2$.
\end{itemize}
\end{ejem}
\newpage
\begin{ejem}\ \\
$\left. \begin{matrix} 2x & - & 4y & + & 6z & = &2\\
 &  & y & + & 2z & = &-3\\
x & + & -3y & + & z & = &4 \end{matrix}\right\}\longrightarrow A^*= \begin{pmatrix} 2&-4&6&|&2\\0&1&2&|&-3\\1&-3&1&|&4\end{pmatrix}$ Calculamos el det. de $A$.\\
$|A|=\deter{2&-4&6\\0&1&2\\1&-3&1}=2+0-8-(6-12+0)= 0\implies \rg(A)<3$. Luego el sistema será o incompatible o compatible indeterminado. Lo haremos por Gauss (también podemos aplicar el método alternativo que vimos en clase que yo denominé ampliar por cuadrados, pero no te lo recomiendo ya que además en este caso el sistema sale indeterminado así que es posible que nos pidan resolverlo y este método sólo nos aporta información de los rangos de $A$ y $A^*$).
\begin{itemize}
	\item Gauss\\
	$\begin{pmatrix} 2&-4&6&|&2\\0&1&2&|&-3\\1&-3&1&|&4 \end{pmatrix}\sim\begin{pmatrix} 1&-2&3&|&1\\0&1&2&|&-3\\1&-3&1&|&4 \end{pmatrix}\sim\begin{pmatrix} 1&-2&3&|&1\\0&1&2&|&-3\\0&-1&-2&|&3\end{pmatrix}\sim\\
	\begin{pmatrix} 1&-2&3&|&1\\0&1&2&|&-3\\0&0&0&|&0 \end{pmatrix}$ Luego es un SCI porque $\rg(A)=rg(A^*)=2<$ nº incógnitas\\
	Y sus soluciones son $\triple{x-2y+3z=1\implies x-2(-3-2\lambda)+3\lambda=1\implies x=-5-7\lambda}{y+2z=-3\implies y+2\lambda=-3\implies y=-3-2\lambda}{z=\lambda}$
\end{itemize}
\end{ejem}
\newpage

\begin{ejem}\ \\
$\left. \begin{matrix} x & +ay & -7z & = &4a-1\\
x& +(1+a)y & -(a+6)z & = &3a+1\\
 & ay & -6z & = &3a-2 \end{matrix}\right\}\longrightarrow A^*= \begin{pmatrix} 1&a&-7&|&4a-1\\1&1+a&-a-6&|&3a+1\\0&a&-6&|&3a-2\end{pmatrix}$ Calculamos el det. de $A$.\\
$|A|=\deter{1&a&-7\\1&1+a&-a-6\\0&a&-6}=-6(1+a)-7a-(a(-a-6)-6a)=-6-6a-7a+a^2+6a+6a=\\=a^2-a-6$. Resolvamos la ecuación de segundo grado y obtenemos las soluciones $a=-2$ y $a=3$, luego det$A=0$ para estos valores de $a$. Ahora distinguimos casos:
\begin{itemize}
	\item $a=-2$\\
	Sabemos que det $A$ es $0$ así que el rango no es $3$. Continuamos por Gauss (sustituyendo $a$ por $-2$):\\
	$\begin{pmatrix} 1&-2&-7&|&-9\\1&-1&-4&|&-5\\0&-2&-6&|&-8\end{pmatrix}\sim \begin{pmatrix} 1&-2&-7&|&-9\\0&1&3&|&4\\0&-2&-6&|&-8\end{pmatrix}\sim\begin{pmatrix} 1&-2&-7&|&-9\\0&1&3&|&4\\0&0&0&|&0\end{pmatrix}$\\
	Tenemos que $\rg{A}=\rg{A^*}=2<$ nº incógnitas, luego SCI. Si nos pide que resolvamos tomamos $z=\lambda$ y resolvemos habitualmente ya que ya tenemos nuestro sistema escalonado.
	\item $a=3$\\
	Sabemos que det $A$ es $0$ así que el rango no es $3$. Continuamos por Gauss (sustituyendo $a$ por $3$):\\
	$\begin{pmatrix} 1&3&-7&|&11\\1&4&-9&|&10\\0&3&-6&|&78\end{pmatrix}\sim\begin{pmatrix} 1&3&-7&|&11\\0&1&-2&|&-1\\0&3&-6&|&7\end{pmatrix}\sim\begin{pmatrix} 1&3&-7&|&11\\0&1&-2&|&-1\\0&0&0&|&10\end{pmatrix}$\\
	Tenemos que $\rg{A}=2\neq3=\rg{A^*}$, luego Sist. Incompatible y no tiene solución.
	\item Si $a\neq -2,3$\\
	En este caso hemos visto generalmente la resolución en función de $a$, es decir, resolvíamos como un sistema compatible determinado normal (por Cramer por ejemplo) y las soluciones $x,y,z$ quedaban en función de $a$ pero esto no es habitual, sino que en un apartado te dicen que lo resuelvas para $a=$ un número, generalmente distinto de los casos anteriores (y por supuesto distinto del caso que dé sistema incompatible en el caso de que lo haya pues este no tiene solución).\\
	Por ejemplo te pueden pedir para $a=-3$ y puede que te pidan un método de resolución específico (Cramer generalmente) o que te dejen hacerlo como tú quieras. Simplemente, como en los casos anteriores sustituimos el valor $a$ por el que corresponda (en este caso $-3$) y resolvemos.\\
	
\end{itemize}

\end{ejem}

\end{document}