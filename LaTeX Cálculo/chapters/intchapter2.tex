\chapter{Teorema de Lebesgue}
	\section{Teorema de Lebesgue y aplicaciones de medibilidad}
	\begin{teor} Teorema de Lebesgue.\\
	Sea $\function{f}{R}{\R}$ acotada, entonces $f$ es integrable de Riemann$\iff D(f)=\{x\in R:f\ \mathrm{no\ es\ continua\ en\ }x\}$ tiene medida nula.
	\end{teor}
	
	\begin{observacion}\ 
	\begin{enumerate}[1)]
	\item  Si $D(f)$ tiene contenido nulo $\implies f$ es integrable.
	\item Las funciones continuas son integrables.
	\item Las funciones continuas excepto en un número finito de puntos son integrables.
	\item Las funciones continuas excepto en una cantidad numerable de puntos son integrables.
	\end{enumerate}	
	\end{observacion}
	
	\begin{corolario} Sea $A\subset\R^n$ acotado, entonces $A$ es medible-Jordan $\iff D(\chi_A)$ tiene medida nula.\\
	En concreto $D(\chi_A)=Fr(A)$, por tanto $A$ es medible-Jordan $\iff Fr(A)$ tiene medida nula $\overset{Fr(A)\ \mathrm{es\ compacto}}\iff Fr(A)$ tiene contenido nulo.
	\end{corolario}
	
	\begin{proposicion} Sea $\function{f}{R}{\R}$ integrable y supongamos que $\{x\in R:f(x)\neq 0\}$ tiene medida nula, entonces $\integral{}{R}f=0$
	\begin{proof}\ \\
	Veamos que $\upint{}{R}f\geq 0$\\
	Sea $P\in \pi(R)$, tenemos que $S(f,P)=\underset{s\in P}\sum\sup\{f(x):x\in S\}v(S)$\\
	$S\nsubset\{x:f(x)\neq 0\}\ \forall S\in P$, pues este último conjunto tiene medida nula y $S$ es un subrectángulo $\implies \exists x\in S\talque f(x)=0\ \forall S\in P\implies \underset{s\in P}\sum\sup\{f(x):x\in S\}v(s)\geq 0\implies\\\implies S(f,P)\geq 0\implies \upint{}{R}f\geq 0$\\
	Análogamente, veamos que $\lowint{}{R}f\leq 0$\\
	Como hemos visto antes, sea $P\in\pi(R),\ \forall S\in P\ \exists x\in S\talque f(x)=0\implies\\\implies \sup\{f(x):x\in S\}v(S)\leq 0\ \forall S\in P\implies s(f,P) \leq 0\implies \lowint{}{R}f\leq 0$\\
	Como $\lowint{}{R}f=\upint{}{R}f\implies \integral{}{R}f=0$
	\end{proof}
	\end{proposicion}
	
	\begin{proposicion} Sea $\function{f}{R}{\R}$ integrable y $f(x)\geq 0\ \forall x\in R$. Supongamos que $\integral{}{R}f=0$, entonces $\{x\in R:f(x)\neq 0\}$ tiene medida nula.
	\begin{proof} \ \\
	Supongamos que $\function{f}{R}{\R}$ es integrable, $f\geq 0$ y $\integral{}{R}f=0$.\\
	Veamos que $\{x\in R:f(x)\neq 0\}$ tiene medida nula.\\
	Tenemos que $\{x\in R:f(x)\neq 0\}=\{x\in R:f(x)>0\}=\stackbin[m=1]{\infty}\bigcup\left\{x\in R:f(x)\geq \dfrac{1}{m}\right\}$\\
	Hemos demostrado que la unión de los elementos de una sucesión de conjuntos de medida nula tiene medida nula, por lo tanto basta con probar que $\left\{x\in R:f(x)\geq \dfrac{1}{m}\right\}$ tiene medida nula $\forall m\in\N$ (de hecho tiene contenido nulo). Veamos esto último:\\
	Sea $A_m=\left\{x\in R:f(x)\geq \dfrac{1}{m}\right\}$ tiene medida nula $\forall m\in\N$, dado $\varepsilon >0$, tenemos\\
	$0=\integral{}{R}f=\upint{}{R}f=\inf\{S(f,P):P\in\pi(R)\}$ por tanto existe $P_0\in \pi(R)\talque\dfrac{\varepsilon}{m}>S(f,P_0)=\underset{S\in P_0}\sum v(S)\sup\{f(x):s\in S\}\geq \underset{S\cap A_m\neq\emptyset}{\underset{S\in P_0}\sum}v(S)\sup\{f(x):x\in S\}\geq\underset{S\cap A_m\neq\emptyset}{\underset{S\in P_0}\sum} v(S)\dfrac{1}{m} $\\ 
	Por tanto: $\underset{S\cap A_m\neq\emptyset}{\underset{S\in P_0}\sum}v(S)<\varepsilon\implies A_m\subset \underset{S\cap A_m\neq\emptyset}{\underset{S\in P_0}\bigcup}S\implies A_m$ tiene contenido nulo$\implies A_m$ tiene medida nula $\forall m\in \N$
	\end{proof}
	\end{proposicion}
	
	\section{Integrabilidad en conjuntos distintos de rectángulos}
	\begin{defi} Sea $A\subset\R^n$ acotado, $\function{f}{A}{\R}$ definimos $\integral{}{A}f=\integral{}{R}\overline{f}$,\\
	donde $\xfunction{\overline{f}}{R}{\R}{x\to\doubleleft{f(x) \mathrm{\ si\ }x\in A}{0 \mathrm{\ si\ }x\notin A}}$ con $R$ rectángulo tal que $A\subset R$. \\
	Entonces tenemos $\integral{}{A}f=\integral{}{R}\chi_A f$
	\end{defi}
	
	\begin{observacion} Sea $\function{f}{R}{\R}$ integrable y sea $A\subset R$ medible-Jordan $\implies f$ es integrable en $A$. La prueba de esto es que $f$ integrable en $A\equiv f\chi_A$ es integrable en $R$. Veamos que $D(f\chi_A)$ tiene medida nula. Es fácil ver que $D(f\chi_A)\subset D(f)\cup Fr(A)$ que tienen medida nula, por tanto $D(f\chi_A)$ tiene medida nula $\implies A$ integrable.	
	\end{observacion}
	
	\section{Propiedades de las integrales}
	
	\begin{proposicion} Linealidad de la integral.\\
	Sea $R\subset \R^n$ rectángulo, sea $A\subset R$ y sea $\function{f,g}{A}{\R}$ integrables entonces $\alpha f+\beta g$ es integrable en $A$ y $\integral{}{A}\alpha f+\beta g=\alpha\integral{}{A}f+\beta\integral{}{A}g$.	
	\begin{proof}\ \\
	Extendamos $f$ y $g$ a $R$ haciéndolas cero fuera de $A$. Tomemos $\sucesion{P}{n}$ sucesión de particiones de $R$ tales que $\norm{P_n}\limited 0$. Ahora, para cualesquiera $x_S\in S$ con $S\in P_n\ \forall n\in\N$ tenemos por el \textit{teorema de Darboux}:
	\begin{center}
	$\integral{}{A}f=\limite{}{\ntiende}\underset{S\in P_n}\sum f(x_S)v(S)$
	\end{center}
	\begin{center}
	$\integral{}{A}g=\limite{}{\ntiende}\underset{S\in P_n}\sum g(x_S)v(S)$
	\end{center}
	Y ahora como $\forall n\in\N$ tenemos que $\underset{S\in P_n}\sum (\alpha f + \beta g)(x_S)v(S)=\underset{S\in P_n}\sum (\alpha f(x_S)+\beta g(x_S))v(S)=\alpha\underset{S\in P_n}\sum f(x_S)v(S)+\beta\underset{S\in P_n}\sum g(x_S)v(S)$\\
	Por tanto $\alpha f+\beta g$ es integrable en $A$ y $\integral{}{A}\alpha f+\beta g=\alpha\integral{}{A}f+\beta\integral{}{A}g$.
	\end{proof}
	\end{proposicion}
	
	\begin{proposicion} Monotonía de la integral.\\	
	Sea $R\subset \R^n$ rectángulo, sea $A\subset R$ y sea $\function{f,g}{A}{\R}$ integrables, si $f\leq g$ entonces $\integral{}{A}f\leq\integral{}{A}g$.
	\begin{proof}\ \\
	Extendamos de la manera habitual $f$ y $g$ a $R$ haciéndola nula fuera de $A$. Tenemos que $g-f\geq 0$. Sea $P\in\pi(R)$, tenemos que $s(g-f,P)\geq 0$.\\
	Por la proposición anterior tenemos que $g-f$ es integrable, por tanto tenemos que $\integral{}{A}(g-f)=\\=\lowint{}{A}(g-f)=\sup\{s(g-f,P):P\in\pi(R)\}$. Como $s(g-f,P)\geq 0\implies\\\implies 0\leq \integral{}{A}(g-f)\overset{\mathrm{por\ prop.\ 1}}=\integral{}{A}g-\integral{}{A}f\implies\ \integral{}{A}f\leq\integral{}{A}g$. 
	\end{proof}
	\end{proposicion}
	
	\begin{proposicion} Aditividad respecto del conjunto de integración.\\
	Sean $A,B$ conjuntos acotados, sea $\function{f}{A\cup B}{\R}$ acotada, y $A\cap B$ medida nula. Si $f$ es integrable en $A$, en $B$ y en $A\cap B$, entonces $f$ es integrable en $A\cup B$ y $\integral{}{A\cup B}f=\integral{}{A}f+\integral{}{B}f$
	\begin{proof}\ \\
	Sea $R$ rectángulo tal que $A\cup B \subset R$, sea $\xfunction{\overline{f}}{R}{\R}{x\to\doubleleft{f(x)\ \mathrm{si\ }x\in A\cup B}{0\ \mathrm{si\ }x\notin A\cup B}}$\\
	$\overline{f}=\overline{f}\chi_{A\cup B}\overset{\chi_A+\chi_B-\chi_{A\cap B}=\chi_{A\cup B}}=\overline{f}\chi_A+\overline{f}\chi_B-\overline{f}\chi_{A\cap B}\implies f$ integrable en $A\cup B$.\\
	Tenemos:\\
	$\integral{}{A\cup B}f=\integral{}{R}\overline{f}\chi_{A\cup B}=\integral{}{R}\overline{f}\chi_A + \integral{}{R}\overline{f}\chi_B -\underset{0\mathrm{\ en\ casi\ todo\ punto}}	{\underset{\shortparallel}{\integral{}{R}\overline{f}\chi_{A\cap B}}}=\integral{}{A}f+\integral{}{B}f$
	\end{proof}
	\end{proposicion}
	
	\begin{observacion} Si $A$ y $B$ son medibles-Jordan, entonces $f$ es integrable en $A\cup B\iff f$ es integrable en $A$ y en $B$, además si esto ocurre, $f$ es integrable en $A\cap B$.
	\end{observacion}
	
	\begin{corolario} Consecuencias inmediatas de las propiedades anteriores.
	\begin{enumerate}[1)]
	\item $f$ integrable en $A\implies |f|$ integrable en $A$ y además $\left|\integral{}{A}f\right|\leq\integral{}{A}|f|$.
	\item Si $f$ es integrable en $A$ medible-Jordan, y $|f(x)|\leq M\ \forall x\in A$, entonces $\left|\integral{}{A}f\right|\leq M\cdot v(A)$.
	\end{enumerate}
	\begin{proof}\ 
	\begin{enumerate}[1)]
	\item $f$ integrable en $A\implies |f|$ integrable en $A$.\\
	Ahora tenemos $-|f|(x)=-|f(x)|\leq f(x)\leq|f(x)|=|f|(x)\ \forall x\in A\ximplies{\mathrm{por\ monotonia}}{} \\
	\implies\integral{}{A}-|f|\leq\integral{}{A}f\leq\integral{}{A}|f|\ximplies{\mathrm{por\ linealidad}}{}-\integral{}{A}|f|\leq\integral{}{A}f\leq\integral{}{A}|f|\implies \left|\integral{}{A}f\right|\leq\integral{}{A}|f|$
	\item Tenemos que $\left|\integral{}{A}f\right|\overset{\mathrm{por\ 1)}}\leq\integral{}{A}|f|\leq \integral{}{A}M\chi_A\overset{\mathrm{por\ linealidad}}=M\integral{}{A}\chi_A=Mv(A)$
	\end{enumerate}
	\end{proof}
	\end{corolario}
	
	\begin{proposicion} Teorema del valor medio para integración.\\
	Sea $A\subset \R^n$ medible-Jordan, conexo y compacto, sea $\function{f}{A}{\R}$ continua, entonces $\exists x_0\in A$ tal que:
	\begin{center}
	$f(x_0)=\dfrac{1}{v(A)}\integral{}{A}f$ $\left(\mathrm{si\ }v(A)\neq 0,\mathrm{\ en\ general\ llegamos\ a\ }f(x_0)v(A)=\integral{}{A}f\right)$.
	\end{center}
	\begin{proof}\ \\
	Por ser $A$ compacto $\implies \exists x_1,x_2\in A \talque f(x_1)\leq f(x)\leq f(x_2)\ \forall x \in A$\\
	Tenemos ahora $f(x_1)v(A)\leq \integral{}{A} f\leq f(x_2)v(A)$\\
	Supongamos $v(A) >0$, entonces $f(x_1)\leq \dfrac{1}{v(A)}\integral{}{A}f\leq f(x_2)\ximplies{f \mathrm{continua\ en\ }A}{A\mathrm{\ conexo}} \exists x_0\in A\talque f(x_0)=\\=\dfrac{1}{v(A)}\integral{}{A}f$.
	Si $v(A)=0$ se cumple trivialmente para cualquier $x_0\in A$
	\end{proof}
	\end{proposicion}
	
	\begin{nota} A $\dfrac{1}{v(A)}\integral{}{A}f$ se le llama ``valor promedio'' o ``valor medio'' de $f$ en $A$.
	\end{nota}