\chapter[Espacios métricos]{Espacios métricos, normados y con producto escalar}
	\section{Espacios métricos}
	
	\begin{defi} 
	 Un espacio m\'etrico $(M,d)$ es un conjunto $M$ con una funci\'on $d \colon M\times M \to \mathbb{R}$ que cumple:
		\begin{enumerate}[1)]
			\item $d(x,y) \geq 0$ $ \forall x, y \in M$
		
			\item $d(x,y) = 0 \Leftrightarrow x = y$
		
			\item $d(x,y) = d(y,x) $ $ \forall x, y \in M$
		
			\item $d(x,y) \leq d(x,z) + d(y,z)$ $    \forall x, y, z \in M$
		\end{enumerate}
	\end{defi}

	\begin{ejem} \ 
		\begin{enumerate}[1)]
			\item En $\R$ con $d(x,y) = |x-y| $ $\forall x, y \in \mathbb{R}$
			\item En $\R ^{n}$ con $d_2(x,y)= \sqrt{\sum_{i = 1}^{n} |x_i-y_i|^{2}} $
			\item En $\R ^{n}$ con $d_1(x,y)= \sum_{i = 1}^{n} |x_i - y_i| $
			\item En $\R ^{n}$ con $d_\infty(x,y)= \underset{1\leq i\leq n}\max|x_i - y_i|$
			\item En $\R^n$ si $p\in (1, +\infty )$ con $d_p(x,y) =(\sum_{i=1}^n |x_i - y_i|)^{1/p}$
			\item En un conjunto arbitrario con
		\[d(x,y)  = \left\{
	       \begin{array}{ll}
		 0      & \mathrm{si\ } x = y \\
		 1 		& \mathrm{si\ } x \neq y \\
	       \end{array}
	     \right.\]
		
			\item En $C\ =\{ f \colon [0,1] \to \mathbb{R}\talque \mathrm{f\ continua\ en\ } [0, 1] \mathrm{con\ } d_\infty(f,g)=\stackbin[{x\in [0,1]}]{}\max |f(x)-g(x)|\}$
			\item Si $M=\{ 0,1\}^n=\{(\theta_1,\theta_2,...\theta_n)\talque\theta_i=1$ \'o $0, \mathrm{\ para\ } 1 \leq i \leq n\} $ con $d((\theta_1...\theta_n),(t_1...t_n))=$ n\'umero de coordenadas distintas.
			\item Sea $(M,d)$ un espacio m\'etrico podemos definir la m\'etrica $p$ que es acotada: $p(x,y)=\\=\dfrac{d(x,y)}{1+d(x,y)} \ \forall x,y \in M$
		\end{enumerate}
	\end{ejem}

		
	\begin{defi}
		 Sea $(M,d)$ un espacio m\'etrico y sea $A \subset M $ definimos el diámetro de $A$ como $\diam(A)=\sup\{d(x,y)\talque x,y \in A\}$
	\end{defi}
	
	\begin{defi}
		Decimos que $A\subset M$ es acotado si tiene di\'ametro finito.\\
	\end{defi}
	
	\begin{ejem} \
		\begin{enumerate}[1)]
			\item $\mathbb{N}$ en $(\mathbb{R},|\cdot|)$ no es acotado.
			\item $\mathbb{N}$ en $(\mathbb{R},\partial)$, siendo $\partial$ la métrica discreta, es acotado.
		\end{enumerate} 
	\end{ejem}
	
	\section{Espacios normados}	
	
	\begin{defi}
		Sea $E$ un espacio vectorial sobre $\mathbb{R}$ una norma en $E$ es una aplicaci\'on \\ $||\cdot||\colon E \to \mathbb{R}$ con las siguienetes propiedades:
		\begin{enumerate}
			\item $\norm{x} \geq 0 \ \forall x \in E$
			\item $\norm{x} = 0 \Leftrightarrow x=0$
			\item $\norm{\lambda x} = |\lambda| \norm{x} \ \forall x\in E, \ \forall \lambda\in E$
			\item $\norm{x+y} \leq \norm{x} + \norm{y} \ \forall x,y \in E$ \textbf{(Desigualdad triangular)}.		
		\end{enumerate}
	\end{defi}
	
	\begin{nota}
	- Un espacio normado $(E,\norm{\cdot})$ es un espacio vectorial junto con la norma $\norm{\cdot}$
	\end{nota}
	
	\begin{ejem} \ 
		\begin{enumerate}[1)]
			\item En $\mathbb{R} ||t||=|t| \ \forall t\in \mathbb{R}$
			\item En $(\mathbb{R}^n,||\cdot||_2)$ donde $||x||_2 =\sqrt{\sum^n_{i=1}x_i^2}\ \forall x \in \mathbb{R}^n$
			\item En $\mathbb{R}^n$ con $||x||_1 =\sum ^n_{i=1}|x_i|$
			\item En $\mathbb{R}^n$ con $||x||_{p_{\ 1<p<+\infty}} =(\sum ^n_{i=1}|x_i|^p)^{1/p}$
			\item En $\mathbb{R}^n$ con $||x||_\infty = \max|x_i|$
			\item En $C([0,1])$ con: 
				\begin{enumerate}[i)]
					\item $||f||_\infty = \max|f(x)| \ \forall f \in C$
					\item $||f||_1 = \int_0^1  |f(x)|dx\ \forall f \in C$
					\item $||f||_2 = \sqrt{\int_0^1  |f(x)|^2dx}\ \forall f \in C$
				\end{enumerate}
		\end{enumerate}		
	\end{ejem}
	
	
	
	\begin{proposicion}
	 Sea $(E,||\cdot||)$ un espacio normado, entonces la aplicación \\
	 $d\colon E\times E \to \mathbb{R}$ definida por $d(x,y)= ||x-y|| \ \forall	x,y \in E$ es una m\'etrica en $E$ \underline{(que llamamos}\\ \underline{ m\'etrica inducida por la norma)}.
	 \end{proposicion}
	
	\section{Espacios vectoriales}	
	
	\begin{defi}
		Sea $E$ un espacio vectorial sobre $\mathbb{R}$, un producto escalar (o producto interno) en $E$ es una aplicaci\'on $\dotproduct{}{}\colon E\times E \to \mathbb{R}$ que cumple:
		\begin{enumerate}[1)]
			\item $\dotproduct{x}{x}\geq 0 \ \forall x\in E$
			\item $<x,x>=0 \Leftrightarrow x=0 \ \forall x \in E$
			\item $<x+y,z>=<x,z>+<y,z> \ \forall x,y,z \in E$
			\item $<\lambda x,y>= \lambda <x,y> \ \forall x,y \in E, \ \forall \lambda \in \mathbb{R}$
			\item $<x,y> = <y,x> \ \forall x,y \in E$
		\end{enumerate}
	\end{defi}
	 
	\begin{ejem} \ 
		\begin{enumerate}[1)]
	 		\item En $\mathbb{R} <t,s>=t\cdot s \ \forall t,s \in \mathbb{R}$
	 		\item En $\mathbb{R}^n <x,y> = \sum^n_{i=1}x_iy_i \ \forall x,y \in \mathbb{R}^n$
	 		\item En $C([0,1]) <f,g>=\int ^1_0 f(x)g(x)dx$
		\end{enumerate}
	\end{ejem}
	
	\begin{corolario} \underline{Consecuencias de la definici\'on}. \\ 
		Si $<,>$ es un producto escalar en un espacio vectorial $E$ se tiene:
		\begin{enumerate}[1)]
			\item $<\lambda x+ \mu y, z>=\lambda <x,z>+\mu <y,z>$
			\item $<x, \lambda y+\mu z> = \lambda <x,y>+\mu <x,z>$
			\item $<x,\lambda y> = \lambda <x,y>$
			\item $<x,0>=<0,x>=0$
		\end{enumerate}
		- $ \forall x,y,z\in E,\ \forall \lambda,\mu \in\mathbb{R}$.
	\end{corolario}
	
	
	\begin{teor}\textbf{Desigualdad de Cauchy - Schwarz.} \\
	Si $<,>$ es un producto escalar en $E$ (sobre $\mathbb{R}$) entonces:
		\begin{align*}
			|<x,y>|\leq \sqrt{<x,x>}\sqrt{<y,y>} \ \forall x,y \in E
		\end{align*}
		\begin{proof}
			Sean $x,y \in E$
			\begin{itemize}
				\item Si $y=0$ \\
		\[\left.
	       \begin{array}{ll}
		 & |<x,y>|=|<x,0>|=0 \\
		 & |<y,y>|=0 
	       \end{array}
	     \right\} \Rightarrow |<x,0>|=\sqrt{<x,x>} \sqrt{0} \] 	
				\item Si $y\neq 0$\\
		Tenemos $<y,y>\ >0$ y $\left(x-\dfrac{<x,y>}{<y,y>} y\right)\in E$ \\
		$0 \leq <x-\dfrac{<x,y>}{<y,y>}y,x-\dfrac{<x,y>}{<y,y>}y>=$ \\
		$= <x,x>-\dfrac{<x,y>}{<y,y>}<x,y> -\dfrac{<x,y>}{<y,y>}<x,y> + \left(\dfrac{<x,y>}{<y,y>}\right)^2<y,y> = $ \\
		$= <x,x> - \dfrac{(<x,y>)^2}{<y,y>} \geq 0 \implies \dfrac{(<x,y>)^2}{<y,y>} \leq <x,x> \implies$ \\
		$\implies(<x,y>)^2\leq <x,x><y,y> \implies |<x,y>|\leq \sqrt{<x,x>}\sqrt{<y,y>}$
			\end{itemize}
		\end{proof}
	\end{teor}	
	
	\begin{observacion}En particular si en $\mathbb{R}^n\ <x,y>=\sum^n_{i=1}x_iy_i$
		\begin{align*}
			\left|\sum^n_{i=1}x_iy_i\right|\leq \sqrt{\sum^n_{i=1}x_i^2}\sqrt{\sum^n_{i=1}y_i^2}
		\end{align*}	
	\end{observacion}
	
	\begin{defi} Si $(E, <,>)$ es un espacio con un producto escalar, definimos en $E$ la norma:
		\begin{align*}
			||x||=\sqrt{<x,x>}
		\end{align*}	
	\end{defi}	
	
	\begin{teor} \textbf{Igualdad del paralelogramo}. \\
		Sea $(E,<,>)$ un espacio vectorial con producto escalar y $||x||=\sqrt{<x,x>}\ \forall x,y\in E$ entonces:
		\begin{align*}
			||x+y||^2+||x-y||^2=2(||x||^2+||y||^2)\ \forall x,y \in E
		\end{align*}	
	\end{teor}
	
	\begin{observacion}No toda norma proviene de un producto escalar, por ejemplo $||\cdot||_\infty$ en $\mathbb{R}^2$
	\end{observacion}

	\begin{proposicion}Sea $E$ un espacio vectorial sobre $\mathbb{R}$ y sea $||\cdot||$ una norma en $E$ entonces $||\cdot||$ proviene de un producto escalar $\iff$ se cumple la igualdad del paralelogramo.
		\begin{proof} \ 
			\begin{itemize}
				\item ($\implies$)  trivial.
				\item ($\impliedby$)  Si definimos $<x,y> = \dfrac{1}{\mu}(||x+y||^2-||x-y||^2)\ \forall x,y 						\in E$ \\
					Se comprueba que $<,>$ es un producto escalar en $E$ y la norma que genera es la 							definida.
			\end{itemize}
		\end{proof}
	\end{proposicion}