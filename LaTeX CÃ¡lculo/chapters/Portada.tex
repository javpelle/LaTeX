\blankpage

\pagenumbering{Roman} % para comenzar la numeracion de paginas en numeros romanos
	\chapter*{}
\begin{flushright}
\textit{Un matemático que no es en algún sentido un\\poeta no será nunca un matemático completo}.\\\ \\ Karl Weierstra\ss
\end{flushright}

	\chapter*{Prefacio}
	El cálculo diferencial es la herramienta del análisis matemático que estudia la transformación de las funciones cuando sus variables cambian. El cálculo integral por su parte nos proporciona las herramientas para la resolución de cálculos infinitesimales y el cálculo de áreas y volúmenes de regiones y sólidos de revolución. El desarrollo de dichas herramientas en este curso nos permitirá, entre otras cosas, resolver problemas de funciones de $\R^n$ en $\R^m$ que se alejan de la asignatura de análisis de variable real de primer curso.\\\\\\
	
	Para cursar estas asignaturas es necesario tener una base en el análisis de una variable. Algunas proposiciones de estos apuntes no están demostradas, bien porque ya que fueron vistas en el curso pasado, o bien porque no han sido demostradas en clase. Pese a ello han sido añadidas algunas de las anteriores en base a otros apuntes y/o bibliografía. Por ello recomiendo revisar los apuntes de David Peñas de \textit{Análisis de variable real} si fuere necesario. \\\\\\
	
	Estos apuntes están basados en las clases de \textit{Cálculo diferencial} de doña María del Pilar Cembranos Díaz y en las clases de \textit{Cálculo integral} de don José Javier de Mendoza Casas durante el curso 2015-2016, en los apuntes de Daniel Azagra de \textit{Cálculo Integral} y en el libro \textit{Marsden, J. and Hoffman, M. (1998). Análisis clásico elemental}. \\\\\\
	
	Por último quiero agradecer a Álvaro Rodríguez por la ayuda prestada con \LaTeX y a Miguel Pascual y Luis Aguirre por aguantarnos los unos a los otros y hacer de esta complicada carrera algo más ameno.
	
	
\mainmatter % Empieza a numerar en números arábigos 

	\tableofcontents