\documentclass[11pt, oneside]{book}


\usepackage[utf8]{inputenc}
\usepackage[spanish]{babel}

%--------Codigos para la caligrafia, tipos de letras%---------------
\usepackage[T1]{fontenc}
%\usepackage[bitstream-charter]{mathdesign} 
%\usepackage{textcomp} %Paquete para algunos caracteres especiales

\usepackage{afterpage}
\usepackage{mathtools}
\usepackage{amssymb}
\usepackage[usenames]{color}
\usepackage{enumerate}
\usepackage{stackrel}
\usepackage{amsthm}
\usepackage{centernot}
\usepackage[a4paper, total={6in, 8in}]{geometry}
\usepackage{pgfplots}
\usepackage[toc,page]{appendix}
\usepackage{graphicx} % graficos

%\renewenvironment{proof}{{\bfseries \ \\\\Demostración}}{\qed} %Demostración en negrita

\newtheorem{teor}{Teorema} % Estilo de texto cursivo

\theoremstyle{definition} % Cambio del estilo de los teoremas a normal
\newtheorem{proposicion}{\textbf{Proposici\'on}}
\newtheorem{defi}{\textbf{Definición}}
\newtheorem{corolario}{Corolario}
\newtheorem{figura}{Figura}
\newtheorem*{lema}{Lema}
\newtheorem*{ejem}{\underline{Ejemplo}}
\newtheorem*{nota}{Nota}
\newtheorem*{observacion}{Observación}

\newcommand{\talque}{\mathrm{\ _\shortparallel\ }}
\newcommand{\R}{\mathbb{R}}
\newcommand{\e}{\mathrm{e}}
\newcommand{\N}{\mathbb{N}}
\newcommand{\Q}{\mathbb{Q}}
\newcommand{\C}{\mathbb{C}}
\newcommand{\Cgot}{\mathcal{C}}
\newcommand{\dom}{\mathrm{Dom\ }}
\newcommand{\gradiente}{\bigtriangledown}
\newcommand{\nimplies}{\centernot\implies}
\newcommand{\nimpliedby}{\centernot\impliedby}
\newcommand{\nlongrightarrow}{\centernot\longrightarrow}
\newcommand{\nsubset}{\centernot\subset}
\newcommand{\oversim}[1]{\stackbin{_\sim}{#1}}
\newcommand{\sucesion}[2]{\left\{#1_{#2}\right\}^\infty_{#2 = 1}}
\newcommand{\sucesionelement}[2]{\left\{#1\right\}^\infty_{#2 = 1}}
\newcommand{\doubleright}[2]{ \left. \begin{array}{ll}	#1 \\	#2 \\	\end{array} 	\right\} }
\newcommand{\doubleleft}[2]{ \left\{\begin{array}{ll}	#1 \\	#2 \\	 \end{array}	\right. }
\newcommand{\doubleleftright}[2]{ \left\{\begin{array}{ll}	& #1 \\	& #2 \\	 \end{array}	\right\}}
\newcommand{\double}[2]{ \left. \begin{array}{ll}	#1 \\	#2 \\	 \end{array}	\right. }	
\newcommand{\tripleright}[3]{ \left. \begin{array}{ll}	#1 \\#2 \\#3\\	\end{array} 	\right\} }
\newcommand{\triple}[3]{ \left. \begin{array}{ll}	#1 \\#2 \\#3\\	\end{array} 	\right. }
\newcommand{\ximplies}[2]{\stackbin[#2]{#1}\implies}
\newcommand{\xiff}[2]{\stackbin[#2]{#1}\iff}
\newcommand{\ximpliedby}[2]{\stackbin[#2]{#1}\impliedby}
\newcommand{\dotproduct}[2]{<#1,#2>}
\newcommand{\norm}[1]{\left|\left|#1\right|\right|}
\newcommand{\function}[3]{#1\colon #2\longrightarrow #3}
\newcommand{\xfunction}[4]{\stackbin[#4]{}{#1\colon #2\longrightarrow #3}}
\newcommand{\limite}[2]{\stackbin[#2]{\ }\lim\ #1}
\newcommand{\limited}{\stackbin{n\rightarrow\infty}\longrightarrow}
\newcommand{\xtiende}{x\rightarrow x_0}
\newcommand{\ntiende}{n\rightarrow\infty}
\newcommand{\ttiende}{t\rightarrow 0}
\newcommand{\htiende}{h\rightarrow 0}
\newcommand{\vacio}{\emptyset}
\newcommand\blankpage{%
    \null
    \thispagestyle{empty}%
    \addtocounter{page}{-1}%
    \newpage}




\title{C\'alculo Integral}


\author{Javier Pellejero Ortega}

\begin{document}

%----------------------------------
%Inicio portada
	\begin{titlepage}
\begin{center}
\vspace*{+0.75in}
\textbf{\textsc{\begin{Huge}Matemática discreta\end{Huge}}}

\vspace*{+0.15in}
\begin{figure}[htb]
\begin{center}
\includegraphics[width=8cm]{../imagenes/complu}\\\ \\\ \\
\textsc{\textbf{\begin{LARGE}Universidad Complutense de Madrid\end{LARGE}}}\\\ \\\ \\\ \\
\textsc{\begin{Large}Facultad de Informática\end{Large}}\\\ \\
\textsc{\begin{large}Doble Grado en Matemáticas e Ingeniería Informática\end{large}}
\end{center}
\end{figure}

\vspace*{0.35in}

\textsc{\textbf{\begin{large} Javier Pellejero \end{large}}\\}
Curso 2014-2015
\end{center}
\end{titlepage}
% Fin portada
%-------------------------------------------------------

\blankpage

\pagenumbering{Roman} % para comenzar la numeracion de paginas en numeros romanos
	\chapter*{}
\begin{flushright}
\textit{Aquí va la dedicatoria y/o quote}
\end{flushright}

	\chapter*{Prefacio}
	Aquí va el prefacio.
	
\mainmatter % Empieza a numerar en números arábigos 

	\tableofcontents
	
	\chapter{Introducción a la matemática discreta y a la lógica matemática}
	\section{Introducción a la lógica Matemática}
	La lógica nos permite representar ideas de manera formal, mediante una sintaxis, que describe la idea y una semántica, que define el significado de la sintaxis.
	La lógica formal se usa entre otras cosas para:
	\begin{itemize}
	\item Formalizar propiedades del mundo.
	\item Interpretar; es decir, asociar a enunciados un significado.
	\item Deducción formal: demostrar que una cierta propiedad es verdadera a partir de unas propiedades anteriores
	\end{itemize}
	Veamos a continuación las aplicaciones de la lógica matemática a la informática:
	\begin{itemize}
	\item Especificación y verificación de programas.
	\item Derivación de programas.
	\item Reducción automática.
	\item Lógica como paradigma de programación: PROGRAMACIÓN LÓGICA.
	\item Inteligencia artificial.
	\end{itemize}
	Formalizaremos oraciones declarativas (verdaderas o falsas), como por ejemplo \textit{el 3 es primo}, o \textit{todos los números racionales son mayores que 7}.\\
	Existen también oraciones no declarativas que no son ni verdaderas ni falsas, como por ejemplo \textit{¿llueve?}.
	\section{Lógica proposicional}
	Representamos un cierto enunciado representado por un símbolo de porposición $(p,q,r,...)$ que puede evaluarse a dos posibles valores (V o F).\\
	Símbolos de proposición:\\
	$\underset{\mathrm{prop.}}p\ |\ \underset{\mathrm{verdad}}T\ |\ \underset{\mathrm{falso}}\perp$
		
	
\end{document}