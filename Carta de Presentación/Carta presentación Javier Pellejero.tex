\documentclass[11pt, oneside]{book}


\usepackage[utf8]{inputenc}
\usepackage[spanish]{babel}

%--------Codigos para la caligrafia, tipos de letras%---------------
\usepackage[T1]{fontenc}
%\usepackage[bitstream-charter]{mathdesign} 
%\usepackage{textcomp} %Paquete para algunos caracteres especiales

\usepackage{afterpage}
\usepackage{mathtools}

\usepackage{amssymb}
\usepackage[usenames]{color}
\usepackage{enumerate}
\usepackage{stackrel}
\usepackage{amsthm}
\usepackage{centernot}
\usepackage[a4paper, total={6in, 8in}]{geometry}

\title{Carta de presentación}


\usepackage{fancyhdr}
\pagestyle{fancy}
\lhead{Carta de presentación}
%\rhead{Javier Pellejero Ortega}

\pagenumbering{gobble}


\author{Javier Pellejero Ortega}


\begin{document}
\ \\
\begin{flushright}
Javier Pellejero Ortega\\
Jacinto Benavente, 10\\
Azuqueca de Henares, 19200, Guadalajara\\
625737110\\
javierpellejer@gmail.com\\
\end{flushright}

Estimados/as señores/as de BBVA.\\

Desde muy joven me han apasionado las matemáticas y el mundo tecnológico, lo cual me llevó a interesarme por multitud de programas informáticos de distintos ámbitos y más tarde a querer crear los míos propios. Todo esto me condujo a querer estudiar el doble grado de Matemáticas e Ingeniería Informática de la Universidad Complutense de Madrid del que actualmente soy alumno de último curso.\\

Durante mi experiencia universitaria he continuado viviendo en mi ciudad natal y esto ha supuesto dedicar un tiempo excesivo al transporte diariamente. Sin embargo, he conseguido compaginar estudios y dichas horas de transporte con clases particulares a diario y la práctica de balonmano. El deporte de equipo ayuda, además de al desarrollo físico, al desarrollo social, a trabajar en equipo o a mejorar a planear tus estrategias ante un reto personal; es por eso que he practicado varios de ellos desde que era niño.\\

En cuanto a mí, a parte de mis habilidades informáticas y matemáticas propias de un estudiante de dichas carreras universitarias, me siento cómodo trabajando en equipo y considero que cuento con capacidad para solventar distintos tipos de problemas en distintos ámbitos. Conozco tanto varios lenguajes de programación como diversas herramientas informáticas, tal y como resalto en mi \textit{CV}, destacando entre ellos \textit{Java} y \textit{C++}.\\

Este año decidí cursar prácticas curriculares a partir de febrero de este curso y para dichas prácticas he querido solicitar un puesto en su empresa. Estoy interesado en multitud de disciplinas tecnológicas y estoy al corriente de que BBVA presta mucho interés en este ámbito contando con un equipo numeroso e instruido. El sector bancario es y ha sido siempre, sin duda, unos de los líderes de la transformación digital y es por eso que a estudiantes como yo les supone un plus de interés dicho sector.\\

Es por eso que solicito esta beca, más allá de realizar unas prácticas para la universidad, quiero adquirir experiencia y conocimiento profesional útil y creo que en su empresa puedo desarrollar no sólo mis habilidades informáticas, sino que también las matemáticas que desde siempre me han gustado.\\

Si quieren más información adicional a la de la presente carta y \textit{CV}, estoy encantado de facilitársela a través de la dirección de correo proporcionada.

\end{document}