\documentclass[11pt, oneside]{book}


\usepackage[utf8]{inputenc}
\usepackage[spanish]{babel}

%--------Codigos para la caligrafia, tipos de letras%---------------
\usepackage[T1]{fontenc}
%\usepackage[bitstream-charter]{mathdesign} 
%\usepackage{textcomp} %Paquete para algunos caracteres especiales

\usepackage{afterpage}
\usepackage{mathtools}

\usepackage{amssymb}
\usepackage[usenames]{color}
\usepackage{enumerate}
\usepackage{stackrel}
\usepackage{amsthm}
\usepackage{centernot}
\usepackage[a4paper, total={6in, 8in}]{geometry}

\title{Carta de presentación}


\usepackage{fancyhdr}
\pagestyle{fancy}
\lhead{Carta de presentación}


\author{Javier Pellejero Ortega}


\begin{document}
\ \\\ \\\ \\
Estimados/as señores/as de BBVA.\\\ \\

Me encuentro acabando mis estudios de Matemáticas e Ingeniería Informática en la Universidad Complutense de Madrid y decidí cursar prácticas curriculares a partir de febrero de este curso\\

Para dichas prácticas he querido solicitar un puesto en su empresa. Estoy interesado en multitud de disciplinas tecnológicas y estoy al corriente de que BBVA presta mucho interés en este ámbito contando con un equipo numeroso e instruido. Es por eso que solicito esta beca, más allá de realizar unas prácticas para la universidad, quiero adquirir experiencia y conocimiento profesional útil y creo que ustedes son una de las mejores opciones para ello.\\

En cuanto a mí, a parte de mis habilidades informáticas y matemáticas propias de un estudiante de dichas carreras universitarias, considero que tengo facilidad para organizar y liderar grupos de trabajo además de ser muy sociable




\end{document}